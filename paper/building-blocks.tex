\section{Building blocks}%
\label{BuildingBlocks}

\dots

\subsection{\Aclp*{KEM}}\label{KEM}

\dots

\subsection{\Acl*{BE}}\label{BE}

We will do onion routing with many alternative nodes per hop.
This means that we encrypt the same message for more than one possible 
recipient.
We will employ \iac{BE} scheme to do this.
More specifically, we are interested in \iac{BE} adapted for \iac{P2P} setting 
where devices can dynamically join and leave, \ie we do not want too much cost 
for joining or leaving, \eg from key regeneration.
We also want some privacy properties, \ie those of \ac{ANOBE}.

\NewScheme{\DBE}{DBE}
\NewAlgorithm{\DBEsetup}{\DBE\method Setup}
\NewAlgorithm{\DBEjoin}{\DBE\method Join}
\NewAlgorithm{\DBEleave}{\DBE\method Leave}
\NewAlgorithm{\DBEenc}{\DBE\method Enc}
\NewAlgorithm{\DBEdec}{\DBE\method Dec}

% XXX Deal with self plagiarism here: \cite{PPACinPubFS}

We need a decentralized \ac{BE} scheme, \(\DBE\).
\(\DBE\) provides the following algorithms:
\DBEsetup,\allowbreak \DBEjoin,\allowbreak \DBEleave,\allowbreak 
\DBEenc,\allowbreak \DBEdec.

The \(\DBEsetup\) algorithm generates the global parameters.
The \(\DBEjoin\) algorithm performs necessary key generation and other 
operations when a device joins whereas \(\DBEleave\) performs the necessary 
operations when a device leaves.
The \(\DBEenc\) algorithm takes a message and a recipient set \(S\) and returns 
a ciphertext that everyone in \(S\) can decrypt using \(\DBEdec\).

\NewScheme{\ANOBE}{ANOBE}
\NewAlgorithm{\ANOBEsetup}{\ANOBE\method Setup}
\NewAlgorithm{\ANOBEkeygen}{\ANOBE\method Keygen}

We can use the \(\ANOBE\) scheme by \textcite{ANOBE} to instantiate our \(\DBE\) 
scheme.
We simply have to break \(\ANOBEsetup\) and \(\ANOBEkeygen\) into 
\(\DBEsetup,\allowbreak \DBEjoin,\allowbreak \DBEleave\).

