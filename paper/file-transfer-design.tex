\subsection{\acs{SPORES}: fragmentation and reassembly across devices}%
\label{SPORES}%
\label{sec:file_transfer}

\sonja{to be merged with the previous subsection}
So far, we have treated the case where Alice sends \emph{one} message to Bob in 
\emph{one} round and any one of Bob's devices receives it.
Now we will consider a slightly different scenario, we will show how to build a 
file transferring service on top of \ac{SPORES}.
Consider that the message \(m\) that Alice wants to send to Bob is too large to 
successfully pass every layer of the route.
\Eg each device goes offline and back online every \(t\) seconds, the bandwidth 
is \(w\) bits per second and the message is \(|m|\) bits; if \(|m| > t w\), 
then the message \(m\) can never be delivered as any device will go offline 
before relaying it to the next node.
The natural solution is to split \(m\) into \(n\) parts, \(m_0, \dotsc, 
  m_{n-1}\), where \(|m|/n\leq t w\).
The problem now is that \(m_i\) and \(m_j\) (\(i\neq j\)) might end up on 
different devices.
The parameters \(t\) and \(w\) must be empirically approximated and used as 
global configuration parameters for the system.
We run simulations for various values of \(t\) and \(w\) in \cref{Performance}.
In this section we describe a file-transfer protocol for splitting a large 
message, sending the fragments using \ac{SPOR} and reassemble them --- even if 
the fragments are received by different devices of the squad.

Alice wants to send the message \(m\) to Bob, \(m\) is \(|m|\) bits long.
She splits \(m\) into \(n\) parts \(m_0, \dotsc, m_{n-1}\).
Alice gives \(|m|\), \(n\), \(\hash(m)\) and \(\hash(m_0), \dotsc, 
  \hash(m_{n-1})\) to Bob over an out-of-band channel.

Bob runs \(\CreateReply\) \(n\) times, the node sets passed to \(\CreateReply\) 
are created as in \cref{SPOR}.
Bob notes the order of the identifiers, \(I\), of each header and sends the 
headers in that order to Alice.

Alice receives the headers and keeps the order of the headers.
For each \(0\leq i < n\), she uses \(\UseReply\) with \(m_i\) as payload to get 
a header \(\hat M_i\).
Then she constructs the final header \(M_i\) by running \(\CreateFwd\) with 
\(\hat M_i\) as payload and the final layer set to \(\{\rdvnode\}\) (to 
indicate to the last node that it is a rendezvous node).
She then sends \(M_i\) along the route.
We note that Alice can process all pieces in parallel as long as she uses the 
headers she got from Bob in the correct order.

When the headers are used in the correct order Bob can use the identifiers from 
the headers (\(I\)) to correctly reassemble the fragments despite out-of-order 
delivery.

Alice also creates reply-headers for Bob, these can be used by Bob to inform 
Alice at regular time intervals about which chunks have been received and which 
are missing.

%\commentDaniel{I leave the remaining details for someone else to write up.
 % What we need is the details of how Bob's devices syncs and reassembles files.
 % Maybe this is reduced to Sprinkler or something.}
