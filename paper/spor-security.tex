\section{\acs*{SPOR} security analysis}%
\label{spor-security}

\paragraph*{Node selection}

\sonja{will motivate using the survey/taxonomy}
The onion route provides unlinkability between Alice and Bob as long as the 
message passes at least one honest node on the route.
This means that any adversary, say Eve,  must ensure that the message passes 
through an adversarial node on each layer.
The probability of successful attack is proportional to Eve's size in the 
network, \ie \(f_A = \frac{|D_A|}{|D|}\).
\commentDaniel{Check that we still use the \(D_i\) notation for the devices in 
  \(i\)'s squad.}
If we assume that \(\GetRandomNode\) returns nodes with close to the uniform 
distribution, then the \(\nu w_L\) nodes selected is expected to contain \(f_A 
  \nu w_L\) nodes controlled by Eve.
If \(f_A \nu w_L < \nu\), then Eve is not expected to be present on every layer 
of the route.
Considering how \ac{SPOR} distributed the \(\nu w_L\) selected nodes, Eve's 
best strategy is to have high-availability nodes --- assume that 
\(\avail_{D_E}(t) = 1\) for all Eve's devices \(D_E\).
This way, Eve's devices will be distributed evenly across all layers (provided 
she controls enough of the selected devices).
To succeed, Eve must be chosen in the first layer, the probability of which 
will be \(\frac{f_A}{w_L}\) (for a passive adversary and \(1\) for an active).
If one of her devices is chosen in the first layer, then she can choose one of 
her own devices in the next layer.
If she is present on every layer, she can ensure she relays the message all the 
way to Bob.
In summary, her probability of success is \(\frac{1}{w_L} (1-(1-f_A)^{(w_L-1) 
    \nu +1})\) (or without the \(\frac{1}{w_L}\) factor for an active 
adversary), which can be compared to \((f_A)^\nu\) for classical \ac{OR} (for 
both active and passive adversaries).
\sonja{compare the chance of being present on every
  layer AND chosen in the first layer to regular onion routing where she
  needs to be the (sole) node on each layer. Are we doing
  better/worse/the same?}
\commentDaniel{Hard to say at this time of night \dots}

However, Eve can only be certain that she has relayed the message from Alice to 
Bob if \(\nu = r\), \ie she receives the message from Alice, relays it \(r-1\) 
times and lastly sends it to Bob.
If \(\nu < r\) though, then, due to the indistinguishability and unlinkability 
properties of SphinxES (inherited from \ac{Sphinx}), Eve cannot know for certain if 
Alice is the first node and Bob is the last node. \sonja{but what can
  she learn/guess?}
\commentDaniel{For all she knows there might be another node at either the 
  beginning or the end, she cannot tell.}


Finally, we must point out that the \(\GetRandomNode\) algorithm must be 
privacy preserving.
Consider the following scenario.
Alice requests a set of peers, say \(\{d_1, \dotsc, d_n\}\).
If \(d_i\) interacts in each sampling operation resulting in \(d_i\) being in 
the sample, then \(d_i\) can generate a new public key \(\pk_d^{(j)}\) for each 
sample \(j\).
This means that, if Alice sampled \(d_i\), she knows the public key 
\(\pk_i^{(j)}\), and if \(d_i\) knows it is included in Alice's sample, then it 
will also know a route belongs to Alice whenever \(\pk_i^{(j)}\) is used to 
encrypt the header.
(This follows from the fact that SphinxES is source-routed.)
\Ie every node on Alice's route would know that Alice is either the source or 
the destination.

Although the requirements of the \ac{RPS} being uniformly randomly
distributed and privacy-preserving are not easily met in practice,
there are some promising candidates~\cite[\eg][]{Octopus,BrahmsRPS}.
\name does not put any other constraints on the \ac{RPS} used.

\paragraph*{Device unlinkability}

Alice transfers a file \(f\) to Bob during the time interval 
\(\interval{t_0}{t_1}\).
We denote the set of Alice's devices as \(D_A\) and Bob's as \(D_B\).
Bob wants to compute an estimate \(\hat D_A\) which is as close to \(D_A\) as 
possible.

At this point, Bob knows that \emph{at least one} of Alice's devices must have 
been online at any given time in \(\interval{t_0}{t_1}\).
We can also assume that Bob knows all data from the \ac{RPS}, \ie the \((d, 
  \pk_d, \avail_d)\) tuple (see \cref{SPOR}) of \emph{every device that was 
  online at some point during \(\interval{t_0}{t_1}\)}.
Let \(\A_{\interval{t_0}{t_1}}\subseteq \N\times \G^*\times (\RR\to 
  \interval{0}{1})\) denote the set of those tuples.
Thus Bob can approximate \(D_A\) by \(\hat D_A \gets 
  \A_{\interval{t_0}{t_1}}\cap D_B\).

When Alice and Bob interact the next time, either Alice sends a file \(f'\) to 
Bob or Bob sends a file \(f'\) to Alice, then Bob can do the same again:
\(\hat D_A' \gets \A_{\interval{t_0'}{t_1'}} \cap D_B\).
If at least one of \(d, \pk_d, \avail_d\) remains constant, Bob can use that to 
improve his approximation.
Say that \(\avail_d\) remains constant.
Then Bob can improve his approximation to \(\hat D_A \cap_{\avail_d} 
  \hat{D}_A'\), where \(\cap_{\avail_d}\) is the intersection which only 
considers the attribute \(\avail_d\).
Even if \(\avail_d\) is not constant, its variation might stay inside some 
\(\epsilon\)-neighbourhood, and if \(\epsilon\) is small enough, \(\avail_d\) 
might still be used to distinguish Alice.
The same applies for \(d\).
(\(\pk_d\) can be regularly updated.)

With an inaccurate availability function, \(\avail\), this will not be a 
problem as there will be many others (as the system scales) that share the 
same availability distribution.
With more accurate availability estimates we would have to employ techniques 
like \ac{DP} to ensure that Alice's availability distribution is not
(too close to)
unique.

\paragraph*{Robustness}

Stateless routing can result in re-ordered chunks or packets, and
chunks received on different devices of a user's \squad. From \ac{Sphinx},
we inherit unique IDs for each mix packet.
If we provide the mix packets as an ordered list when we communicate them 
out-of-band together with the metadata of the file and they are used in that 
same order, then we can use those IDs to re-order chunks if need be.
The \squad overlay handles intra-squad device 
coordination.
We ensure liveness by acknowledgments and timeouts (as detailed in 
\cref{sec:squad_overlay}), despite the possibility of \squad devices
going offline during transfer or other node or link failures.


\paragraph*{Extended adversary model}%
\label{security-limitations}

A global, passive, network-observing adversary can of course do
statistical disclosure attacks to eventually learn who communicates
with whom~\cite{StatisticalDisclosureAttacks}. There are also denial
of service attacks that can lead to the adversary learning more from
the system~\cite{DenialOfSecurity}.  As mentioned in
\cref{system-model}, we share the adversary model with \ac{Tor}
%and Sphinx
and thus exclude a global passive adversary. Our focus is on
protecting Alice and Bob from each other and from others who also
participate as nodes in the network, not on protecting them from
surveillance. Nevertheless, we offer some reflections on such an
adversary. We hypothesize that our stateless and redundant design
might yield some advantages over classical \ac{OR}. For
example, no control messages are sent back to the source if the
adversary has introduced an error.  \sonja{``But these advantages will
  only postpone the success of the attack, eventually even our
  protocol must resend.'' why?}  \commentDaniel{Due to statistics. We should 
  also look into
  this denial of security.} Another example is that the source-bounded
random-walk and new routes for file chunks make tracking more
difficult than with regular onion routing circuits. A drawback, however, is
that since the chunks can take several paths from source to destination, the source 
will send and the destination will receive more traffic than other nodes in the 
network~\cite{RoutingSurveyAnonymousProtocols}. However, this is mitigated by the fact that the source and 
destination can be several different devices --- \ie from the \squads.


We would also like to point out that the
vulnerability to a global passive attacker is not inherent in Sphinx
or SphinxES, but can be caused by the way they are applied, \ie the
traffic sent. If performance and resource consumption were not an
issue, batched and well-timed message or file transfer by all nodes
would render traffic analysis less efficient. Concretely, \Ac{SPORES} inherits 
the security properties from SphinxES and \ac{SPOR}.
However, both SphinxES and \ac{SPOR} can be used with any mixing strategy, and 
the security they provide is as secure as the mixing strategy (or flushing 
algorithm) allows.
Since \ac{SPORES} is a fully-fledged system, it must make a concrete choice for 
the mixing strategy.
Since \ac{SPORES} aims at low latency, routers do not batch and mix incoming 
messages; they immediately forward any messages.
This puts \ac{SPORES} in the same position as \ac{Tor}, \ie more vulnerable to 
traffic analysis than batching mix-nets~\cite{RoutingSurveyAnonymousProtocols} 
\commentDaniel{we should probably look for the original papers} --- this 
vulnerability is inherited from the mixing strategy.


% \commentDaniel{Predecessor attacks: random walk protocols has the problem that 
%   the adversary can see the initiator (statistically) more often than the other 
%   nodes, and can thus identify the 
%   source~\cite{RoutingSurveyAnonymousProtocols}.
%   Our protocol is a \enquote{bounded} random walk, \ie within limits.}



