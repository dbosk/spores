\section{Conclusion and Future Work}

We presented \(\SPOR\) that, in analogy to their biological
counterparts, enables dissemination under unfavorable conditions. More
specifially, the deliberate dissemination of files without using
storage on third-party devices, be they peers or cloud service
providers. This constraint results in the unfavorable conditions of
reduced a-priori availability and therefore unreliable nodes. We build
resilience to these conditions by using predictive modeling for user
behavior and the ensuing availability of their devices. From our
experiments, we find that even suboptimal prediction already results in
successful file transfer with high probability, though often at low
speed. More in-depth knowledge of the users' behavior would allow us
to predict device connectivity farther into the future and thus likely
improve the throughput. From a security and privacy perspective, we
were guided by the properties of TOR~\cite{Tor}, which we modified to
a stateless protocol to accommodate unreliable nodes. While we took
the first steps toward analyzing the consequences of the modifications
on security and privacy, more is needed to ensure that the added
features of unreliable nodes, peer sampling, and the side effect of
enabling more local TOR routing (for example, keeping routes inside a
country) does not compromise security. On a related note, it would be
interesting to adapt the Sphinx~\cite{Sphinx} cryptographic packet
format to the routing of \(\SPOR\) as a step toward provable security.


% \commentAL{To rewrite as real paragraphs} Future work and prospects:
% \begin{itemize}
% 	\item More in-depth knowledge of the user's behavior would allow to predict devices connections farther in the future;
% 	\item Changing devices' asymmetric keys regularly would increase security.
% 	Updating keys is easy in \name given the RPS that propagates new information to all nodes.
% 	However, more reflection would be needed to avoid key updates breaking ongoing communications passing through the devices in question.

% \end{itemize}

% It would be interesting to adapt the Sphinx~\cite{Sphinx} cryptographic packet 
% format to the routing of \(\SPOR\).
% Then we could achieve the properties of Sphinx and enjoy provable security.
