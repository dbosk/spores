\section{Conclusion and Future Work}

\commentAL{To rewrite as real paragraphs} Future work and prospects:
\begin{itemize}
	\item More in-depth knowledge of the user's behavior would allow to predict devices connections farther in the future;
	\item Changing devices' asymmetric keys regularly would increase security.
	Updating keys is easy in \name given the RPS that propagates new information to all nodes.
	However, more reflection would be needed to avoid key updates breaking ongoing communications passing through the devices in question.

\end{itemize}

We see two alternative paths for improving the header.
First, we might be able to adapt the Sphinx~\cite{Sphinx} cryptographic packet 
format to the routing of \(\SPOR\).
Then we could achieve the properties of Sphinx and enjoy provable security.

The second approach would be to use the technique by \textcite{PPACinPubFS}.
They adapt \(\ANOBE\) in such a way that its ciphertexts decrypt to different 
things for different recipients.
This way we could have one ciphertext encrypted for all the alternative nodes 
--- across layers.
Then the ciphertext would decrypt to different plaintexts for nodes on different 
layers in the route.
This would hide the path length (constant-size ciphertexts by padding) and the 
position of individual nodes.
This approach would probably be the easiest, but would probably not provide all 
the properties from Sphinx (\eg resistance to active tagging attacks and bitwise 
unlinkability).

We should provide bitwise unlinkability for the message between hops on the 
route.
Currently the message (\(m\) in \cref{SPORFwd}) itself remains constant 
throughout the route and thus constitutes a tag for tracking.
This problem can be solved by choosing the encryption scheme \(\Enc*\) (used to 
encrypt the message in \cref{sec:file_exchange}) to have ciphertexts that can be 
re-randomized, \eg schemes for universal 
re-encryption~\cite{UniversalReencryption}.
However, then we must assume that at least one node is honest and actually 
performs the re-randomization (re-encryption) --- there is no incentive for them 
to do this, as neither Alice nor Bob can detect it without talking directly to 
each other to compare the ciphertexts.
