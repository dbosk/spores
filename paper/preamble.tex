\usepackage[binary-units]{siunitx}
\usepackage[british]{babel}
\usepackage[hidelinks]{hyperref}
\usepackage{algorithm}
\usepackage[noend]{algpseudocode}
\usepackage[single]{acro}
\usepackage{booktabs}
\usepackage[T1]{fontenc}
\usepackage[unq]{unique}
\usepackage[utf8]{inputenc}
\usepackage{amsthm}
\usepackage{booktabs}
\usepackage{kbordermatrix}% .sty in paper dir
\usepackage{listings}
\usepackage{mathtools}
\usepackage{newclude}
%\usepackage{subcaption}
\usepackage{thmtools}
\usepackage{xfrac}
\usepackage{xparse}
\usepackage{xspace}
\usepackage[inline]{enumitem}

\usepackage[all]{foreign}
\renewcommand{\foreignfullfont}{}
\renewcommand{\foreignabbrfont}{}

\usepackage[strict]{csquotes}

%% From IEEE template
%\usepackage{cite}
\usepackage{amsmath,amssymb,amsfonts}
%\usepackage{algorithmic}
\usepackage{graphicx}
\usepackage{textcomp}
\def\BibTeX{{\rm B\kern-.05em{\sc i\kern-.025em b}\kern-.08em
    T\kern-.1667em\lower.7ex\hbox{E}\kern-.125emX}}


\DeclareMathOperator{\powerset}{\mathcal{P}}

\lstset{%
  basicstyle=\footnotesize,
  numbers=left
}
\let\email\texttt


% Adrien: What is the following for?
\usepackage[noamsthm,notheorems]{beamerarticle}
\declaretheorem[numbered=unless unique,style=theorem]{theorem}
\declaretheorem[numbered=unless unique,style=definition]{definition}
\declaretheorem[numbered=unless unique,style=definition]{assumption}
\declaretheorem[numbered=unless unique,style=definition]{protocol}
\declaretheorem[numbered=unless unique,style=example]{example}
%\declaretheorem[style=definition,numbered=unless unique,
%  name=Example,refname={example,examples}]{example}
\declaretheorem[numbered=unless unique,style=remark]{remark}
\declaretheorem[numbered=unless unique,style=remark]{idea}
\declaretheorem[numbered=unless unique,style=exercise]{exercise}
\declaretheorem[numbered=unless unique,style=exercise]{question}
\declaretheorem[numbered=unless unique,style=solution]{solution}

\usepackage[sets]{cryptocode}

\usepackage[natbib,style=ieee]{biblatex}
\usepackage[capitalize]{cleveref}
\usepackage{bibsp}
\addbibresource{meta.bib}
\addbibresource{osn.bib}
\addbibresource{anon.bib}
\addbibresource{be.bib}
\addbibresource{crypto.bib}
\addbibresource{p2p.bib}
\addbibresource{otrmsg.bib}
\addbibresource{ac.bib}
\addbibresource{p2p-private-cloud.bib}
\addbibresource{adrien.bib}

\DeclareAcronym{OR}{%
  short = {OR},
  long = {Onion Routing},
}
\DeclareAcronym{POR}{%
  short = {POR},
  long = {Probabilistic Onion Routing},
}
\DeclareAcronym{SPOR}{%
  short = {SPOR},
  long = {Stateless Predictive Probabilistic Onion Routing},
  short-format = \scshape
}
\DeclareAcronym{SPORES}{%
  short = {SPORES},
  long = {\it Stateless Predictive Onion Routing for E-Squads},
  short-format = \scshape
}
\DeclareAcronym{RPS}{%
  short = {RPS},
  long = {random-peer sampling},
}

\newcommand{\name}{\ac{SPORES}\xspace}
\newcommand{\squad}{e-squad\xspace}
\newcommand{\squads}{e-squads\xspace}
\newcommand{\Squad}{E-squad\xspace}
\newcommand{\Squads}{E-squads\xspace}

% sphinxes.tex
\NewVariable{\sk}{x}
\NewVariable{\pk}{X}
\NewAlgorithm{\Enc}{Enc}
\NewScheme{\Sphinxes}{Sphinxes}
\NewAlgorithm{\CreateHeader}{\Sphinxes[CreateHeader]}
\NewAlgorithm{\CreateFwd}{\Sphinxes[CreateForward]}
\NewAlgorithm{\CreateReply}{\Sphinxes[CreateReply]}
\NewAlgorithm{\UseReply}{\Sphinxes[UseReply]}
\NewAlgorithm{\ProcessHeader}{\Sphinxes[ProcessHeader]}
\NewVariable{\G}{\mathcal{G}}
\NewVariable*{\mac}{\mu}
\NewVariable*{\prg}{\rho}
\NewVariable*{\prp}{\pi}
\NewVariable{\hash}{h}
%\NewVariable{\tag}{\tau}
\NewVariable{\secret}{s}
\NewVariable{\blind}{b}
\NewVariable*{\N}{\mathcal{N}}
\NewVariable*{\D}{\mathcal{D}}
\NewVariable*{\nullnode}{\varepsilon}
\NewVariable{\rdvnode}{*}
\NewAlgorithm{\ComputeKeys}{ComputeKeys}
\NewAlgorithm{\ComputePadding}{ComputePadding}
\NewVariable{\ProcessMessage}{ProcessMessage}

%spor.tex
\NewAlgorithm{\GetRandomNode}{GetRandomNode}
\NewVariable{\avail}{avail}




\usepackage{breqn}

\usepackage{import} % for pdf+latex
%\graphicspath{{figures/}}

\usepackage{ifthen}
\usepackage{calc}
\usepackage{color}
%\usepackage{tikz}

\setlength{\marginparwidth}{1.2cm}

\newcommand{\annote}[3]{{%
		\colorbox{#3}{\bfseries\sffamily\footnotesize\textcolor{white}{#2}}
		\color{#3}\footnotesize
		% \ifthenelse{\equal{#1}{}}{[\scshape #2]}{
		$\blacktriangleright$\textit{#1}$\blacktriangleleft$}
}

\newcommand{\ftanote}[2]{\marginpar{\centering #1 \fbox{#2}}}
\DeclareRobustCommand{\change}[3]{{\color{#3}#1\ftanote{\color{#3} \sc}{#2}}}

% \newcommand{\fixme}[1]{\annote{#1}{Fixme}{red}}
% \newcommand{\flaw}[1]{\annote{#1}{flaw}{red}}
% \newcommand{\todo}[1]{\annote{#1}{Todo}{red}}
% \newcommand{\done}[1]{\annote{#1}{Done}{OliveGreen}}
% \newcommand{\ongoing}[1]{\annote{#1}{OnG}{orange}}
% \newcommand{\maybe}[1]{\annote{#1}{Maybe}{ProcessBlue}}
% \newcommand{\niceToHave}[1]{\annote{#1}{NiceToHave}{ProcessBlue}}
% \newcommand{\tweak}[1]{\annote{#1}{Tweak}{magenta}}
% \newcommand{\revise}[1]{\annote{#1}{Revise}{magenta}}
% \newcommand{\alt}[1]{\annote{#1}{Alt}{magenta}}
% \newcommand{\somenote}[1]{\annote{#1}{Note}{OliveGreen}}
% \newcommand{\more}[1]{\annote{#1}{More}{RubineRed}}
% \newcommand{\add}[1]{\annote{#1}{Add}{RubineRed}}
% \newcommand{\addRef}[1]{\annote{#1}{Ref}{red}}
% \newcommand{\criticism}[1]{\annote{#1}{Criticism}{RubineRed}}
% \newcommand{\answer}[1]{\annote{#1}{Answer}{RubineRed}}

\newcommand{\commentDaniel}[1]{\annote{#1}{Daniel}{magenta}}
\newcommand{\commentAL}[1]{\annote{#1}{AL}{blue}}
\newcommand{\david}[1]{\annote{#1}{David}{red}}
\newcommand{\commentFT}[1]{\annote{#1}{FT}{brown}}
\newcommand{\changeFT}[1]{\change{#1}{FT}{brown}}
\newcommand{\sonja}[1]{\annote{#1}{Sonja}{violet}}

%% Uncomment for final version
% \renewcommand{\change}[3]{#1}
% \renewcommand{\annote}[3]{}



%% \newcommand{\annote}[3]{{\color{#3}%
%% 	   %% \ifthenelse{\equal{#1}{}}{[\scshape #2]}{
%%        $\blacktriangleright$\textsc{#2:} {\em #1}$\blacktriangleleft$} %% \normalcolor
%%   %% }
%% }

%\newcommand{\annote}[3]{{\color{#3}%
%	   \ifthenelse{\equal{#1}{}}{\scshape #2}{[\textsc{#2:} \emph{#1}]} \normalcolor}}
%\renewcommand{\annote}[3]{}

