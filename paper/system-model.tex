\section{System model, adversary model and desired privacy properties}%
\label{system-model}

\name is a distributed system made of users' devices. 
Each device can communicate with any other device that it knows the
address for (\eg using the Internet). All devices belonging to one user
 know which other devices belong to the same user, \ie the \squad.
Thus the devices from an \squad only knows the mapping from devices to users for the devices in 
the \squad it belongs to.
We also assume that the users can exchange some limited amounts of data 
out-of-band, \ie through some other channel than the network just mentioned (\eg 
using \ac{NFC}).

\paragraph*{In the cloud setting} if Alice wants to share a file with
Bob, she picks up and pushes one of her files, from one of her devices, to the cloud
provider, to make it available to Bob that will thereafter get it from
one of its devices. The use of the cloud provider as a third party
provides two main privacy properties, which we have to consider when
changing the setting to that of \ac{P2P}~\cite{DevilInMetadata}:
\begin{itemize}
\item Alice and Bob are the only ones who learn 
about Alice sharing a file with Bob, except for the cloud provider.
\item Alice and Bob learn nothing about each others' devices, \ie 
they cannot map any device as belonging to each other, except for the
cloud provider that learn it.
\end{itemize} 

\paragraph*{In the \name setting} when changing from the cloud setting
to \ac{P2P}, we want to keep the privacy that Alice and Bob already
enjoyed through the cloud provider, and go a step further by removing 
cloud provider that may act as a spy. However, this increased privacy
should not be done to the benefit of the user's friends that may
become the most powerful potential attackers, and act
as an adversary type of friends.
In this context, in \name, we want to protect users from friendly
surveillance~\cite{FriendlySurveillance} in an honest-but-curious setting.

However, we will not consider a global network adversary, only subsets of colluding 
parties. We know that it is notoriously difficult to achieve security against a global 
network adversary~\cite{SystemsForAnonymousCommunication}.
