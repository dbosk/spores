\section{System model, adversary model and desired privacy properties}%
\label{system-model}

\name is a distributed system made of users' devices. 
Each device can communicate with any other device that it knows the
address for (\eg using the Internet). All devices belonging to one user
 know which other devices belong to the same user, \ie the \squad.
Thus, the devices from an \squad only know the mapping from devices to users for the devices in 
the \squad they belong to.
We also assume that users can exchange some limited amounts of data 
out-of-band, \ie through some other channel than the network just mentioned (\eg 
using \ac{NFC} or QR codes).

%system model: explain e-squad, contrast with cloud

%adversary model: remove provider, compensate, results in same as in sphinxOR/Tor?  friends?

%properties: cam/lys: correctness, integrity, security (wrap, indistinguishability) - do not want friends to learn 1) which devices are ours 2)  their availability (assumption: RPS, not lookup)

We first describe a simple model of data sharing via cloud storage to
highlight desirable privacy properties and then map them to our provider-less
setting. We then add privacy properties and adversary models for onion
routing to cover network traffic analysis.

\paragraph*{In the cloud setting} if Alice wants to share a file with
Bob, she picks up and pushes one of her files, from one of her devices, to the cloud
provider, to make it available to Bob to download to
one of his devices. The use of the cloud provider, as a third party,
provides two main privacy properties, which we have to consider when
changing the setting to that of \ac{P2P}~\cite{DevilInMetadata}:
\begin{itemize}
\item Alice and Bob are the only ones \sonja{narrow down to users?} who learn 
about Alice sharing a file with Bob, except for the cloud provider.
\item Alice and Bob learn nothing about each others' devices, \ie 
they cannot map any device as belonging to each other or draw
inferences from their behavior. The cloud
provider, however, can learn this. \sonja{expand to other users?}
\end{itemize} 

\paragraph*{In the \ac{P2P} setting} we want to keep the privacy that
Alice and Bob already enjoyed thanks to the cloud provider functioning
as a gatekeeper toward each other and other users. We go a step
further by removing the cloud provider that may draw inferences from
metadata even when the data itself is encrypted. Removing the provider
also removes the gatekeeping which needs to be compensated for to
protect users from friendly surveillance~\cite{FriendlySurveillance}
by peers (other \name users including Alice and Bob)\sonja{removed: in
  an honest-but-curious setting}. We give the definitions for the
security and privacy properties of \name in Section~\ref{security-discussion}. 

\paragraph*{In the \name setting} we combine the requirements from
the \ac{P2P} setting with those from onion routing for anonymity.
Camenisch and Lysianskaya \cite{} define security properties for onion
routing: correctness, integrity, security (wrap resistance,
indistinguishability). These properties cover the concerns about
revealing to others that Alice shares a file with Bob in the first
place. In \name, we additionally derive the property of
device-unlinkability from the \ac{P2P} requirements, meaning that
using \name does not reveal the \squad a device belongs to nor the
owner and can thus not infer anything about a user's behavior such as
being online.

\paragraph*{Adversary model} we share the model by Tor \cite{} and
Sphinx \cite{}: the adversary has complete control over some part of
the network as well as some of the nodes. We not consider a passive
global network adversary, against which it is notoriously difficult to
achieve security~\cite{SystemsForAnonymousCommunication}.
