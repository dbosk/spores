\section{User model} 
\label{sec:user_model}

We consider many users owning several devices, that they freely turn on and off depending on their location and will.
The participants' devices are the only nodes constituting our system; the unreliability of their connectivity has to be taken for granted.

We will first present a model of each user's behavior, that will drive the connection intervals of our system's nodes.

\subsection{A model of the user's behavior}
\label{sub:a_model_of_the_user_s_behavior}

In \name, we take interest in users that own several devices, and wish to privately exchange files with each other.
To the best of our knowledge, no real-world traces of multi-devices usage over several days exist, as would be needed for the experimental evaluation of our system.
For that reason, we propose a user behavioral model having the following features:

\begin{itemize}
	\item Each user owns a variable amount of devices, that can be either mobile or immobile;
	\item The user travels between three ``locations'' according to a simple probabilistic model: her home, her workplace, and outside;
	\item She can act her devices only when she is close to them. For instance, a user cannot shut down her workstation while being home or outside.
\end{itemize}

To this end, we use a Hidden Markov Model (HMM). 
The \emph{hidden} Markovian process dictates the user's progression among the three proposed locations.
The \emph{observable} process emits ``toggle'' actions on the user's devices, based on probabilities associated with each location.
For instance, the probability that the user switches her home computer on or off (i.e. \emph{toggles}) is null while she is at work or outside.

Such a statistical model is overly simplistic (e.g. it does not encapsulate the time or the weekday), but it is on purpose: simplicity is bliss.
Given the lack of real-world data for our testbed, building a more complex and realistic user model would add no value to our proposal: 
our goal is to see devices exhibiting frequent disconnections and reconnections (or \emph{churn}).
Our simple model being able to put extravagant stress on our system, it is sufficient to access \name's resilience.