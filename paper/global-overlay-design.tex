\subsection{The Global Overlay}
\label{sec:global_overlay}

\newcommand{\viewsize}{\ensuremath{l_{\text{view}}}}
\NewAlgorithm{\GetRandomPeer}{RPS()}


To reach this goal, we employ Random Peer Sampling (RPS) \cite{Voulgaris_Gavidia_van_Steen_2005,Jelasity_Voulgaris_Guerraoui_Kermarrec_van_Steen_2007}.
Essentially, each node maintains a \emph{view} containing \viewsize other devices' descriptors. 
By periodically exchanging random descriptors with a carefully selected node from its view, RPS allows devices to always have a fairly random sample of the nodes in the system inside their view.
Due to the RPS's eviction policy, stale descriptors are removed quickly: each device thus knows a random sample of \viewsize connected nodes.
An RPS descriptor for device $d$ is constituted of the following information:
\begin{itemize}
  \item Its address: $@_d$;
  \item Its public key: \(\pk_d\);
  \item Its probability of being online in the near future, as provided by the Squad Overlay: $p_d=P\left[d \in O_{t+1} | S_t\right]$.
\end{itemize}

Devices now have all the information they need to establish reliable PORs between two sets of devices.



% \commentDaniel{%
%   What is the probability distribution of the peer sample that we get?
%   Will it be close to uniformly random?
%   I need this for the security analysis.%
% }

% \NewAlgorithm{\GetRandomPeer}{GetRandomPeer}

% Every time a device is online, it participates in the random peer sampling 
% protocol.
% If a device is sampled it provides the following information:
% \begin{itemize}
%   \item its address, $@_d$;
%   \item its public key, \(\pk_d\);
%   \item the probability of being online, $p_d=P\left[d \in O_{t+1} | S_t\right] $.
% \end{itemize}
% We will denote this by the following algorithm $(@_d, \pk_d, p_d)\gets 
%   \GetRandomPeer$, which will be used below.
% The probability of being online, \(p_d\), is inferred as above 
% (\ref{sub:a_model_of_the_user_s_behavior}).

