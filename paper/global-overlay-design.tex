\subsection{The Global Overlay}
\label{sec:global_overlay}

\newcommand{\viewsize}{\ensuremath{l_{\text{view}}\xspace}}
\newcommand{\view}{\ensuremath{\mathcal{V}_{\text{RPS}}\xspace}}
\NewAlgorithm{\GetRandomPeer}{random descriptor from \view}

To create PORs, devices need to know the address, encryption key, and probability of remaining available of some nodes in the system, to pick them as relays on the route.

To reach this goal, we employ Random Peer Sampling (RPS) \cite{Voulgaris_Gavidia_van_Steen_2005,Jelasity_Voulgaris_Guerraoui_Kermarrec_van_Steen_2007}.
Essentially, each node maintains a view \view containing \viewsize other devices' descriptors. 
This view is periodically updated as follows: a device $d$ pops the oldest descriptor $d'$ from its view, then swaps a predefined number of $l$ elements from \view with $d'$.
Both devices add a descriptor for themselves to the view exchange.
If $d'$ was offline, its descriptor is simply removed from the $d$'s view, but \view is otherwise not updated.

This allows for two things: firstly, each device's view contains a constantly changing random sample of the whole set of devices;
secondly, offline nodes get removed from one's view fairly fast---we can say that a \view contains \emph{mostly} descriptors of online devices.

In \name, an RPS descriptor for device $d$ is constituted of the following information:
\begin{itemize}
  \item Its address: $d$;
  \item Its public key: \(\pk_d\);
  \item Its probability of being online in the near future, as provided by the Squad Overlay: $p_d=P\left[d \in O_{t+1} | S_t\right]$.
\end{itemize}

Devices now have all the information they need to establish reliable PORs.

% \commentDaniel{%
%   What is the probability distribution of the peer sample that we get?
%   Will it be close to uniformly random?
%   I need this for the security analysis.%
% }

% \NewAlgorithm{\GetRandomPeer}{GetRandomPeer}

% Every time a device is online, it participates in the random peer sampling 
% protocol.
% If a device is sampled it provides the following information:
% \begin{itemize}
%   \item its address, $@_d$;
%   \item its public key, \(\pk_d\);
%   \item the probability of being online, $p_d=P\left[d \in O_{t+1} | S_t\right] $.
% \end{itemize}
% We will denote this by the following algorithm $(@_d, \pk_d, p_d)\gets 
%   \GetRandomPeer$, which will be used below.
% The probability of being online, \(p_d\), is inferred as above 
% (\ref{sub:a_model_of_the_user_s_behavior}).

