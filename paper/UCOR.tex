\section{\Acladj*{UC} \acl*{OR}}%
\label{UCOR}
The \ac{UC} framework is a mathematical framework that can be summarised in
two ideas, first, that there exists a stringent and well defined idea
about what a protocol needs to do, its functionality. Secondly, that
any two protocols that fulfil this idea should be {\it
  indistinguishable\/} in the sense that they should perform the exact
same service~\cite{UniversalComposability}. If the protocol is regarded as a black box, this means
that any input into the two different boxes should result in the
exact same output.

The second idea also means that any two protocols, with well defined
functionalities, should not interfere with one another if they are
are used in parallel or one after the other, as long as their inputs
are correct. That is, they are composable~\cite{UniversalComposability}.

The underlying framework is quite technical and requires one to handle
a computational model with what the original author %TODO: Insert reference
called Interactive Turing Machines and on top of this designed the
Universal Composability theorem \cite{UniversalComposability}. However, the most important part of
this framework is what is known as an \emph{ideal functionality}, which
can be thought of as a black box that fulfils some well-defined
functionality\cite{UniversalComposability}. The idea behind the \ac{UC} framework is then that we ought
to define a protocol which we must then prove simulates the ideal
functionality, so that distinguishing between the ideal functionality
and the real protocol becomes infeasible.

However, this is not necessarily the way one must go about working in
the framework. We will, for example, instead utilise the fact that one
may prove some sufficient conditions that if fulfilled means that the
protocol must simulate the ideal functionality.

One  successful use of the \ac{UC} framework is a model
created by Jan Camenish and Anna Lysyanskaya (a model we will call the
CL-model). This model is an attempt at formalising the concept of
onion routing using the UC-framework. It relies on an ideal
functionality, and five properties that ensures a protocol would properly
simulate the ideal functionality~\cite{CL-model}. We will however not
use these five original properties, rather opting to use three newer
properties~\cite{kuhn}.

We begin by describing the ideal functionality, and it can be
described as consisting of two parts, the internal data structure and
the set of instructions in the form of messages. We begin with the
data structure, describing what parties and information we have.

\subsubsection*{Ideal Onion Routing Functionality: Internal Data Structure.}
\begin{itemize}
\item{The set {\it Bad\/} of  parties controlled by the adversary \(S\).}
  
\item{An onion \(O\) is stored in the a tuple on the form \((sid, P_s,
  P_r, m, n, \mathcal{P}, i)\) where: \(sid\) is the
  identifier,\(P_s\)is the sender, \(P_r\) is the recipient, \(m\) is
  the message sent through the onion routers, \(n < N\) is the length
  of the onion path, \(\mathcal{P} (P_{o_1},\ldots, P_{o_n})\) is the
  path over which the message is sent (by convention,\(P_{o_0}=P_s\),
  and \(P_{o_{n+1}}=P_r\)),\(i\) indicates how much of the path the
  message has already traversed (\(i= 0\) when the onion is at
  \(P_s\)). We know that the onion has reached its destination when
  \(i=n+ 1\).
}
    
\item{ A list \(L\) of onions that the adversarial routers are
  currently processing. Each entry of the list is in the form of a
  tuple \((temp, O, j)\), where \(temp\) is the temporary id provided
  by the ideal functionality, while \(O= (sid, P_s, P_r, m, n,
  \mathcal{P}, i) \) is the onion itself, and \(j\) is the entry in
  \(\mathcal{P}\) that indicates where the onion should go next (the
  adversary does not get to see \(O\)and \(j\)). Remark: Entries will
  never be removed from \(L\). This allows us to model the replay
  attack for the ideal adversary.
}
      
\item{ For each honest party \(P_i\), there is a buffer \(B_i\) of
  onions that are currently being held by \(P_i\). Each entry consists
  of \((temp', O)\), where \(temp'\) is the temporary id that the
  ideal functionality provides, so that the router may process the
  onion and \(O = (sid, P_s, P_r, m, n, P, i)\) is the onion itself
  (\(O\) is hidden from the honest party). Entries are removed from
  this buffer if an honest party informs the ideal functionality that
  it wishes to send the corresponding onion to the next party.
}

\end{itemize}

Every party gets its instructions by way of messages sent between
them, and for this we define 4 messages.

The parties involved receives their instructions from messages that
they send between them, and the ideal functionality have four
messages. 

\subsubsection*{Ideal Onion Routing Functionality: Instructions.}

\begin{itemize}

\item{({\texttt Process_New_Onion}, \(P_r, m, n, \mathcal{P}\)). When
  receiving a message of this form from \(P_s\), where \(m\in
  \{0,1\}^{l_m}\cup\{\bot\}\), do.}
  \begin{enumerate}
  \item{If \(|\mathcal{P}| \geq N\), reject.}
  \item{Otherwise, create a new session id \(sid\), and let \(O =
    (sid, P, P_r, m, n, \mathcal{P}, 0)\). Send the message
    ({\texttt Process_Next_Step}, \(O\)) to itself.}
  \end{enumerate}

\item{({\texttt Process_Next_Step}, \(O\)). Suppose \(O = (sid, P_s,
  P_r, m, n, \mathcal{P}, i)\). The ideal functionality now looks at
  the next part of the path. The router \(P_{o_i}\) is passing the
  onion to the router \(P_{o_{i+1}}\). Depending on which routers are
  honest and which ones are adversarial, two different steps may be
  taken, an honest and an adversarial one:

  \begin{enumerate}
    
  \item{{\bf Honest next}.if the next node, \(P_{o_{i+1}}\), is
    honest.Here, the ideal functionality creates a new random temporary id
    \(temp\) for this onion and sends it to \(S\) (since \(S\) controls
    the network, it decides which messages get delivered): ``Onion
    \(temp\) from \(P_{o_i}\) to \(P_{o_{i+1}}\)''. It adds the entry
    \((temp, O, i+1)\) to the list \(L\). (See ({\texttt Deliver_Message},
    \(temp\)) for the next step.)}

  \item{{\bf Adversary next}.Suppose that \(P_{o_{i+1}}\)is
    adversarial. Then there are two cases:
    \begin{itemize}
    \item{There is an honest router remaining on the path to the
      recipient. Let \(P_{o_j}\)be the next honest router. In this
      case, the ideal functionality creates a random temporary id
      \(temp\) for this onion, and sends the message ``Onion \(temp\)
      from \(P_{o_i}\), routed through \((P_{o_{i+1}}, \ldots,
      P_{o_{j-1}})\) to \(P_{o_j}\)'' to the ideal adversary \(S\),
      and stores \((temp, O, j) \) in the list \(L\).}
      
    \item{ \(P_{o_i}\)is the last honest router on the path; in
      particular, this means that \(P_r\) is adversarial as well. In
      this case, the ideal functionality sends the message``Onion from
      \(P_{o_i}\) with message \(m\) for \(P_r\) routed through
      \((P_{o_{i+1}}, \ldots,P_{o_n})\)'' to the adversary \(S\).}
    \end{itemize}
  }    
  \end{enumerate}
}

\item{({\texttt Deliver_Message}, \(temp\)). This is a message of
  confirmation from
  \(S\) that it sends to the ideal process to notify it that it agrees that
  the onion with temporary id \(temp\) should be delivered to its
  intended destination. To process this message, the functionality
  checks if the temporary identifier \(temp\) corresponds to any onion
  \(O\) on the list \(L\). If it does, it retrieves the corresponding
  record\((temp, O, j)\) and update the onion: if \(O= (sid, P_s, P_r,
  m, n, \mathcal{P}, i)\), it replaces \(i\) with \(j\) to indicate
  that we have reached the \(j\)'th router on the path of this
  onion. If \(j < n+ 1\), it generates a temporary identifier
  \(temp'\), sends ``Onion \(temp'\) received'' to party \(P_{o_j}\),
  and stores the resulting pair(\(temp', O= (sid, P_s, P_r, m, n,
  \mathcal{P}, j)\)) in the buffer \(B_{o_j}\) of party
  \(P_{o_j}\). Otherwise, \(j=n+ 1\), so the onion has reached its
  destination: if \(m \neq \bot\)it sends ``Message \(m\) received''
  to router \(P_r\): otherwise no message is delivered.
}

\item{({\texttt Forward_Onion}, \(temp'\)). This is a message from an
  honest ideal router \(P_i\) notifying the ideal process that it is
  ready to send the onion with id \(temp'\) to the next hop. In
  response, the ideal functionality
  
  \begin{itemize}
    
  \item{Checks if the temporary identifier \(temp'\) corresponds to
    any entry in \(B_i\). If it does, it retrieves the corresponding
    record \((temp', O)\).
  }

  \item{ Sends itself the message ({\texttt Process_Next_Step},
    \(O\)).
  }

  \item{ Removes \((temp', O)\) from \(B_i\).
  }
  \end{itemize}
}
\end{itemize}


To prove that a protocol properly simulates this ideal functionality,
the CL-model, as mentioned above, defined five sufficient properties~\cite{CL-model}.
However, later research has found that there exists three properties
that provide the same thing, making the model slightly easier to
describe~\cite{kuhn}.

\begin{definition}{Onion-Correctness.}
  Let \((G, \text{FormOnion}, \text{ProcOnion})\) be an Onion Routing
  scheme with maximal path length \(N\). Then for all polynomial
  numbers of routers \(P_i\), for all settings of the public
  parameters \(p\), for all \((PK(P), SK(P))\) generated by
  \(G(1^\lambda, p, P)\) for all \(n < N\), for all messages \(m \in
  \mathcal{M}\) and for all onions \(O_1\) formed as \((O_1, \ldots,
  O_{n+1}) \leftarrow
  \text{FormOnion}(m,(P_1,\ldots,P_{n+1}),(PK(P_1),\ldots,PK(P_{n+1})))\)
  the following is true:

  \begin{enumerate}
    \item{Correct path: \(\Pr[\mathcal{P}(O_1, P_1) = (P_1, \ldots,
      P_{n+1})] \geq 1 - negl(\lambda).\)}
    \item{Correct layering: \(\Pr[\mathcal{L}(O_1, P_1) = (O_1, \ldots,
        O_{n+1})] \geq 1 - negl(\lambda).\)}
    \item{Correct decryption: \(\Pr[(m,\bot) =
        \text{ProcOnion}(SK(P_{n+1}), O_{n+1}, P_{n+1}) \geq 1 -
        negl(\lambda). ]\)}
  \end{enumerate}
\end{definition}

\begin{definition}{Tail-Indistinguishability TI.}
  \begin{enumerate}
    \item{The adversary receives as input the challenge public key
      \(PK\), chosen by the challenger by letting \((PK,SK) \leftarrow
      G(1^\lambda, p, P_j)\) and the router name \(P_j\).}
    \item{The adversary may submit any number of onions \(O_i\) of her
      choice to the challenger. The challenger sends the output of
      \(\text{ProcOnion}(SK, O_i, P_j)\) to the adversary.}
    \item{The adversary submits a message \(m\), a path \(\mathcal{P}
      = (P_1, \ldots, P_j, \ldots, P_{n+1})\) with the honest node at
      position \(j, 1\leq j \leq n+1\) of her choice and key pairs for
      all nodes \((PK_i,SK_i)\) (\(1\leq i \leq n+1\) for the nodes on
      the path and \(n+1\) for the other relays).}
    \item{The challenger checks that the router names are valid, that
      the public keys correspond to the secret keys and that the same
      key pair is chose if the router names are equal, and if so, sets
      \(PK_j = PK\) and sets bit \(b\) at random.}
    \item{The challenger create the onion with the adversary's input
      choice:
      \[ (O_1, \ldots, O_{n+1}) \leftarrow \text{FormOnion}(m,
      \mathcal{P}, (PK)_{\mathcal{P}})
      \]
      and a random onion with a randomly chosen path
      \(\bar{\mathcal{P}} = (\bar{P}_1, \ldots, \bar{P}_k = P_j,
      \ldots, \bar{P}_{n+1} = P_{n+1})\), that includes the subpath
      from the honest relay to the corrupted receiver starting at
      position \(k\) ending at \(\bar{n} +1 \):
      \[ (\bar{O}_1, \ldots, \bar{O}_{n+1} \leftarrow
      \text{FormOnion}(m, \bar{\mathcal{P}}, (PK)_{\mathcal{P}})).
      \]}
    \item{
      \begin{itemize}
        \item{If \(b = 0\), the challenger gives \((O_{j+1},
          P_{j+1})\) to the adversary.}
        \item{Otherwise, the challenger gives, \((\bar{O}_{k+1},
          \bar{P}_{j+1})\) to the adversary.}
      \end{itemize}
    }
    \item{ The adversary may submit any number of onions \(O_i\) of
      her choice to the challenger. The challenger sends the output of
      \(\text{ProcOnion}(SK,O_i, P_j)\) to the adversary.}
    \item{The adversary produces guess \(b'\).}
  \end{enumerate}
  TI is achieved if any PPT adversary \(\mathcal{A}\) cannot guess
  \(b' = b\) with a probability non-neglibly better than \(\frac{1}{2}\).
\end{definition}

\begin{definition}{Layer-Unlinkability LU}
  \begin{enumerate}
  \item - 4 are the same as in the definition of TI.
    \setcounter{enumi}{4}
  \item{The challenger creates the onion with the adversary's input
    choice:
    \[(O_1, \ldots, O_{n+1} \leftarrow \text{FormOnion}(m,
    \mathcal{P}, (PK)_{\mathcal{P}}))    
    \]
    and a random onion with a randomly chosen path \(\bar{\mathcal{P}}
    = (\bar{P}_1, \ldots, \bar{P}_k = P_1, \ldots, \bar{P}_{k+j} =
    P_j, \bar{P}_{k+j+1}, \ldots, \bar{P}_{n+1})\), that includes the
    subpath from the honest node of \(\mathcal{P}\) starting at
    position \(k\) ending at \(k+j\) (with \(1 \leq j+k \leq \bar{n}
    +1 \leq N\)), and a random message \(m' \in \mathcal{M}\):
    \[ (\bar{O}_1, \ldots, \bar{O}_{\bar{n} + 1}) \leftarrow
    \text{FormOnion}(m', \bar{\mathcal{P}}, (PK)_{\bar{\mathcal{P}}}).
    \]
  }
  \item{
    \begin{itemize}
      \item{If \(b = 0\), the challenger gives \((O_1,
        \text{ProcOnion(O_j)})\) to the adversary.}
      \item{Otherwise, the challenger gives \((\bar{O}_k,
        \text{ProcOnion(O_j)})\) to the adversary.}
    \end{itemize}
  }
  \item{The adversary may submit any number of onions \(O_i,\, O_i
    \neq O_j,\, O_i \neq \bar{O}_{k+j}\) of her choice to the
    challenger. The challenger sends the output of
    \(\text{ProcOnion}(SK, O_i, P_j)\) to the adversary.}
  \item{The adversary produces guess \(b'\).}
  \end{enumerate}
  LU is achieved if any PPT adversary \(\mathcal{A}\) cannot guess
  \(b' = b\) with a probability non-negligibly better than \(\frac{1}{2}\).
\end{definition}

With these properties and the SphinxES protocol, we wish to prove that
SphinxES does indeed have these properties, and thus fulfils the
security properties of the ideal functionality. 
