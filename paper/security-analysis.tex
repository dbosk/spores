\section{Security Discussion}%
\label{SecurityDiscussion}

Alice wants to prevent Bob from learning which are her devices.
Bob is allowed to infer the behaviour of devices in the Global Device Retinue, 
but a file transfer from Alice should not give Bob any information about which 
devices are Alice's.

Bob constructs \iac{SPOR} route that he gives to Alice.
This means that Alice does not learn any of Bob's destination devices.
When Alice starts sending the message, she will not connect directly to the 
device at the head of the route, since that might be Bob's device.
Alice will construct her own \ac{SPOR} route.
Thus Alice's devices are hidden from Bob's observation and vice versa.

Alice and Bob also want to prevent Carol and David to learn that Alice sent a 
message to Bob.
Alice and Bob construct their \ac{SPOR} route by including devices from the 
Global Device Retinue, which also includes devices from Carol and David.
If all devices on the route belong to Carol and David, then they can, of course, 
learn that Alice sent a message to Bob.
How well they can do if they do not control the entire route, depends on the 
choice of the encryption scheme \(\Enc*\), used to encrypt the message \(m\) in 
in \cref{sec:file_exchange}.

What makes sense is to choose at least an authenticated encryption 
scheme~\cite{AuthEncryption}.
(We could also choose a deniable encryption scheme~\cite{DeniableEncryption}, 
\eg the one by \textcite{OTPKX} which is both deniable and authenticated and 
provides \ac{DEN-SS}.)
This would give confidentiality (or deniability) and integrity for the message, 
but if the first and the last node belong to Carol and David, they can infer 
that Alice sent a message to Bob.
Note that as our in-package header is designed, each node on the route can infer 
its position on the route \emph{relative to the destination}.
(We discuss how to remedy this below.)
None of the nodes can know if they are the first node or not, unless they 
collude with the previous node.
The last node, however, can know for sure that the next hop is the destination, 
but not who it belongs to.

If we choose \(\Enc*\) as a universal re-encryption~\cite{UniversalReencryption} 
scheme (possibly composed with the deniable, authenticated encryption above), 
then we could also provide bitwise unlinkability for the message between hops on 
the route.
(We would also need to modify \(\SPOR[\Fwd]\), \cref{SPORFwd}, to perform the 
re-encryption of \(m\) before forwarding the message to the next hop.)
Without universal re-encryption, the message \(m\) itself remains constant 
throughout the route and thus constitutes a tag for tracking.
As long as \emph{at least one node on the route is honest}, \ie actually 
performs the re-encryption --- there is no incentive for them to do this, as 
neither Alice nor Bob can detect cheating without talking directly to each other 
to compare the ciphertexts --- then the link between Alice and Bob is broken.
This means that Carol and David must control all nodes on the route to link 
Alice's transfer to Bob.

Since Alice and Bob choose a new key for every file, we have forward secrecy 
between files.
\Eg an adversary who compromises either Alice or Bob cannot determine what Alice 
sent to Bob, unless it is an ongoing transfer.


\subsection{Limitations}
from mails:  - Route fingerprinting in anonymous communications.

It depends on the random peer sampling I suppose. And I have no clue 
about how that one works. Adrien referred to a paper by Anne-Marie.

If I were to guess, I'd say: if Anne-Marie's RPS protocol is Byzantine 
fault tolerant, maybe; otherwise, we're most likely screwed.

> - Bridging and fingerprinting: Epistemic attacks on route selection

I'd say the same as above. It's about peer selection.

> - Breaking the collusion detection mechanism of morphmix

We don't do any collusion detection, so I'd say we're vulnerable. We 
assume we choose nodes uniformly at random, so colluding nodes have 
uniform probability of success --- but that falls back on the RPS and 
Anne-Marie's paper. So maybe or screwed.

> - Denial of service or denial of security? How attacks on reliability can
> compromise anonymity

After just reading the abstract, most likely vulnerable. But thanks to 
our stateless design we don't have to do retransmission the whole way, 
as might be the case for active circuits --- only the node who 
experience the failure going forward will need to retransmit. (Thanks 
Adrien for insisting on creating a route back for the ACK ;-)) So we 
might not be in as bad a position as other systems, it'll take more 
rounds of attack to break it --- but I'm quite sure it will break.

But in general, these all face global and possibly malicious 
adversaries. We don't have that in our adversary model. If everyone 
colludes against Alice and Bob in our setting, they're screwed.

It's basically impossible to remain anonymous against the global 
adversary (statistical inference, paper by Danezis) unless Alice sends a 
message to everyone whenever she wants to send a message to Bob. 
Basically the anonymous communication community is converging on methods 
like Loopix and HORNET --- have enough throughput so that the global 
adversary is overwhelmed with data. (Tor is too slow.)

\subsection{Future Work}
% \commentAL{To rewrite as real paragraphs} Future work and prospects:
% \begin{itemize}
% 	\item More in-depth knowledge of the user's behavior would allow to predict devices connections farther in the future;
% 	\item Changing devices' asymmetric keys regularly would increase security.
% 	Updating keys is easy in \name given the RPS that propagates new information to all nodes.
% 	However, more reflection would be needed to avoid key updates breaking ongoing communications passing through the devices in question;
% 	\item \name is not resilient to Byzantine nodes for now: malicious users could potentially make users pick Predictive Onion Routes that pass through them with a high probability, thus allowing the attacker to learn the file sender and receiver. To circumvent this, we would need to employ a secure Random Peer Sampling \cite{Jesi_Montresor_van_Steen_2010}, and prevent devices to lie on their probability of being online.

% \end{itemize}

We see two alternative paths for improving the header.
First, we might be able to adapt the Sphinx~\cite{Sphinx} cryptographic packet 
format to the routing of \(\SPOR\).
Then we could achieve the properties of Sphinx and enjoy provable security.

The second approach would be to use the technique by \textcite{PPACinPubFS}.
They adapt \(\ANOBE\) in such a way that its ciphertexts decrypt to different 
things for different recipients.
This way we could have one ciphertext encrypted for all the alternative nodes 
--- across layers.
Then the ciphertext would decrypt to different plaintexts for nodes on different 
layers in the route.
This would hide the path length (constant-size ciphertexts by padding) and the 
position of individual nodes.
This approach would probably be the easiest, but would probably not provide all 
the properties from Sphinx (\eg resistance to active tagging attacks and bitwise 
unlinkability).