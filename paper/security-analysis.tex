\section{Security analysis}%
\label{SecurityAnalysis}

Alice wants to prevent Bob from learning which are her devices.
Bob is allowed to infer the behaviour of devices in the Global Device Retinue, 
but a file transfer from Alice should not give Bob any information about which 
devices are Alice's.

Bob constructs \iac{SPOR} route that he gives to Alice.
This means that Alice does not learn any of Bob's destination devices.
When Alice starts sending the message, she will not connect directly to the 
device at the head of the route, since that might be Bob's device.
Alice will construct her own \ac{SPOR} route.
Thus Alice's devices are hidden from Bob's observation and vice versa.

Alice and Bob also want to prevent Carol and David to learn that Alice sent a 
message to Bob.
Alice and Bob construct their \ac{SPOR} route by including devices from the 
Global Device Retinue, which also includes devices from Carol and David.
If all devices on the route belong to Carol and David, then they can, of course, 
learn that Alice sent a message to Bob.
How well they can do if they do not control the entire route, depends on the 
choice of the encryption scheme \(\Enc*\), used to encrypt the message \(m\) in 
in \cref{sec:file_exchange}.

What makes sense is to choose at least an authenticated encryption 
scheme~\cite{AuthEncryption}.
(We could also choose a deniable encryption scheme~\cite{DeniableEncryption}, 
\eg the one by \textcite{OTPKX} which is both deniable and authenticated and 
provides \ac{DEN-SS}.)
This would give confidentiality (or deniability) and integrity for the message, 
but if the first and the last node belong to Carol and David, they can infer 
that Alice sent a message to Bob.
Note that as our in-package header is designed, each node on the route can infer 
its position on the route \emph{relative to the destination}.
(We discuss how to remedy this in \cref{conclusion}.)
None of the nodes can know if they are the first node or not, unless they 
collude with the previous node.
The last node, however, can know for sure that the next hop is the destination, 
but not who it belongs to.

If we choose \(\Enc*\) as a universal re-encryption~\cite{UniversalReencryption} 
scheme (possibly composed with the deniable, authenticated encryption above), 
then we could also provide bitwise unlinkability for the message between hops on 
the route.
(We would also need to modify \(\SPOR[\Fwd]\), \cref{SPORFwd}, to perform the 
re-encryption of \(m\) before forwarding the message to the next hop.)
Without universal re-encryption, the message \(m\) itself remains constant 
throughout the route and thus constitutes a tag for tracking.
As long as \emph{at least one node on the route is honest}, \ie actually 
performs the re-encryption --- there is no incentive for them to do this, as 
neither Alice nor Bob can detect cheating without talking directly to each other 
to compare the ciphertexts --- then the link between Alice and Bob is broken.
This means that Carol and David must control all nodes on the route to link 
Alice's transfer to Bob.

Since Alice and Bob choose a new key for every file, we have forward secrecy 
between files.
\Eg an adversary who compromises either Alice or Bob cannot determine what Alice 
sent to Bob, unless it is an ongoing transfer.

