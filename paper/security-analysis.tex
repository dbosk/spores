\section{Security analysis}%
\label{SecurityAnalysis}

\commentDaniel{Under editing.
  Must move suggestions for fixes to future work.}

Alice wants to prevent Bob from learning which are her devices. Bob is allowed 
to infer the behaviour of devices in the global device swarm, a file transfer 
from Alice should not give Bob any information about which devices are Alice's.

We also prevent Carol and David to know that Alice and Bob exchanged a file. 
They forwarded unknown messages.

The message \(m\) remains the same from hop to hop during the onion routing, 
hence a global observer can track the message through the network.
The same applies if one node occurs more than once in a route, \eg in both 
Alice's and Bob's part of the route.

Since we have an in-package header, each node on the route can infer its 
position on the route.

Since we choose a new key for every file, we have forward secrecy between files.
\Eg an adversary who compromises either Alice or Bob cannot determine what Alice 
sent to Bob.

We can choose different properties for the encryption scheme, \(\Enc*\), used to 
encrypt the message \(m\) in in \cref{sec:file_exchange}, depending on what 
security properties we would like to achieve.
What makes sense is to choose at least an authenticated encryption 
scheme~\cite{AuthEncryption}.
This would give confidentiality and integrity for the message, but if the first 
and the last node on the route colludes, they can infer that Alice sent a 
message to Bob.
We could also choose a deniable encryption scheme~\cite{DeniableEncryption}, \eg 
the one by \textcite{OTPKX} which is both deniable and authenticated and 
provides \ac{DEN-SS}.

If we choose \(\Enc*\) as a universal re-encryption~\cite{UniversalReencryption} 
scheme (possibly composed with the deniable, authenticated encryption above), 
then we could also provide bitwise unlinkability for the message between hops on 
the route.
(We would also need to modify \(\SPOR[\Fwd]\), \cref{SPORFwd}, to perform the 
re-encryption of \(m\) before forwarding the message to the next hop.)
Without universal re-encryption, the message \(m\) itself remains constant 
throughout the route and thus constitutes a tag for tracking.
As long as \emph{at least one node on the route is honest} and actually performs 
the re-encryption --- there is no incentive for them to do this, as neither 
Alice nor Bob can detect it without talking directly to each other to compare 
the ciphertexts --- then the link between Alice and Bob is broken.
(Alice and Bob are, of course, still susceptible to traffic analysis.)

