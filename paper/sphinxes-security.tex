\section{SphinxES security analysis}%
\label{SphinxES-security}

There are two changes that we have made that must be analysed.
We changed the key agreement from a pure \ac{DHKE}, \ie based on the \ac{DH} 
assumption, to one containing an indirection.
We also changed from the classical \ac{OR} of a fixed path with one node per 
layer to a set of alternative nodes per layer.

The first change:
The confidentiality of the key previously rested entirely on the \ac{DH} 
assumption (\(\alpha\) in the header).
Now it also rests on the security of the \ac{PRP} \(\prp'\) (\(C\) in the 
header).
The integrity is still reduced to the security of the \ac{MAC} \(\mac\) 
(\(\gamma\) in the header).
We still achieve the bitwise unlinkability since \(C\) is replaced at each 
layer and the next \(C\) is hidden inside the header --- which otherwise 
remains identical to that of \ac{Sphinx}.

The second change is that we use several nodes per layer as opposed to only 
one.
The proofs of \ac{Sphinx} rest on the results of \textcite{CLOnionRouting}, which 
are results for classical \ac{OR} with only one node per layer.

Analogously to classical \ac{OR}, we also reveal the previous and next sets of 
nodes to the adversary. We note that this is actually quite natural as a 
malicious adversary can drop any incoming traffic to a router and wait for the 
timeout and
retransmission, \ie when the 
node has selected a new node from the next layer and retried sending.
As such, the adversary can learn all nodes in the layer --- just as for 
classical \ac{OR}, regardless of whether the set making up a layer contains a
single or several nodes.

With these arguments, we believe that SphinxES inherits the \ac{UC} security 
properties of \ac{Sphinx}. %, however, this should be formally analysed.

