\section{Introduction}

% What's the problem?
Following the Weiserian vision, nowadays, we are
evolving in a new era of cooperation and interactions among humans.
The exponential proliferation of devices connected to the Internet is
drastically impacting human experience and enhancing human
interactions \cite{Dearman:2008, Oh:2017, Sohn:2008,Harper08}. It further opens new area of research such as
\textit{proxemics} applied to ubiquitous computing
\cite{Marquardt:2011}. Users are evolving in their social space with 
multiple kinds of computing devices that form a \textit{squad} of users'
devices, named \squad in the remainder of the
paper. Leveraging on the help of their \squad, users increase
their expectations in terms of connectivity and interactions
possibilities. For instance, as users bring
their devices close to one another, they expect to initiate cross
device interactions \textit{in situ} to share digital content
\cite{Oh:2017} among devices of their own \squad or with the ones 
from other users' \squad. More broadly, users expect to share digital content,
whatever the content location, from one
device to another in a seamless manner \cite{Dearman:2008}, as they
would have done in the old days with an USB key or with an email. 

From a technical perspective, one major trend to provide a seamless
sharing of digital contents among users' devices is to leverage on the
cloud, which acts as a key enabler for ubiquitous computing. Users' data
are pushed online, into the cloud with either Dropbox, Google Docs,
OneDrive, Amazon Drive, etc. to make it accessible anytime,
anywhere to the users' devices.  In the last decades, a lot of research work has
emerged to promote the use of this offloading strategy to overcome scarce
resources of users' devices, and to provide an always-on availability
\cite{Zhang:2014,Gordon:2012,Chun:2011}. It paves the way of different kind
of clouds, such as mobile cloud computing, mobile-edge computing
and/or fog computing, that aim to use the nodes from the cloud
infrastructure that are the closest to end users
for offloading with an improved quality of service and user experience.

Hence, the cloud, and its variants constitute a perfect complement
to perform storage and computation on behalf of the resource
constrained devices. However, this cloud reliance comes at a strong price:
devices become totally tethered to a cloud ecosystem and provider. As
a direct consequence, users lose the control over their own data as it is
stored and spanned into multiple foreign data centers. Hence, it becomes difficult at 
the legal level to assess the privacy, and
confidentiality delivered to the end users. Further, at the practical
level, users' data can be harvested on their behalf, and without their
consent, by either the cloud provider or third parties.
Data sent by the user can be encrypted, but it is not enough
\cite{granick_2017, HooffLZZ15, HarnikPS10}.  As stated by a famous quotation from
NSA: "we kill people based on metadata''. Metadata such as the users'
location and activity can be still tracked, including the access
control history revealing what users share with whom and when.
Users’ devices become tracking and/or monitoring \squad that turn to
be the best spy over users. 

Constructing a system that provides anonymous communication is a very
challenging issue. The last decades, we have witnessed a
 huge set of attempts at building privacy preserving and anonymous p2p
 networks \cite{Clarke:2001, Gnunet2002, Freedman:2002, Nambiar:2006,
   Rennhard:2002}. They are mainly based on onion
 routing concepts \cite{Chaum:1981}, with some improvements to perform node
 discovery in a distributed manner, which are otherwise
centralized in traditional onion routing schemes. However, due to high
churn of peer-to-peer networks, performance of these systems is
discouraging \cite{LeBlond:2013}; churn implies frequent
reconstruction of routes, which is very expensive. In comparison, Tor
\cite{Dingledine:2004}, one of the most popular anonymous network
mitigate lack of performances of anonymity at the cost of a
centralized design through the use of directory
servers to advertise available trusted nodes to bootstrap Onion
routes. 

In this paper, we introduce \name, Stateless Predictive Onion Routing
for \squad, a novel approach to perform a decentralized, autonomous,
and self-organizing file sharing among users' devices while preserving the users’ privacy
in terms of data and metadata. We leverage on the latest works achieved
in p2p systems, privacy and anonymity. The originality of our approach come
from two key aspects. First, we consider proxemics relationships as a key enabler to initiate file
transfers. Intimate distance among users enable them to share
routing information out-of-band, e.g. from their
\squad. Compared to the state of the art on p2p systems, with \name,
users agree out-of-band to share a file but
do not keep it available for anyone else in the future. Second, we
rely on distributed machine learning technics to predict the
availability of nodes. As such, \name provides optimized onion routes; it avoids routes
reconstruction under churns.

Our contributions are as follows: \david{please update}
\begin{itemize}
\item We showed the effectiveness of \iac{ANOBE} scheme by \textcite{ANOBE} 
  to build a \ac{DeBE} as required by \name to
  provide.... 
\item A \squad overlay 
\item Model the user’s behavior as a Hidden Markov Models (HMMs)
\item SPOR provides stateless algorithms for the onion-routing.
\item An evaluation testbed where each user’s behavior
\end{itemize}

In the following, we detail \name 's concepts and approach...

% Cashmere: resilient anonymous routing.
% Route fingerprinting in anonymous communications.
% Bridging and fingerprinting:
% Epistemic attacks on route selection
% Breaking the collusion detection mechanism of morphmix
% Denial of service or denial of security? How attacks
% on reliability can compromise anonymity