% -*- tex-main-file: "contents.tex" compile-command: "pdflatex p2p-private-cloud.tex" ispell-dictionary: "american" -*-
%!TEX root = p2p-private-cloud.tex

\section{Introduction}

% What's the problem?

Fulfilling Mark Weiser's vision, we now rely on a increasing number of personal digital devices for many everyday tasks. These devices form a \textit{personal squad} of connected entities (or \emph{\squad} as we shall call them in the remainder of the
paper), that users rely on for their everyday interactions with their fellow humans, as well as with the material and digital worlds.
The growing importance of these \squad{}s is profoundly modifying how we live~\cite{Dearman:2008, Oh:2017, Sohn:2008,Harper08}, and has opened new areas of research such as \textit{proxemics}, the study of proximity-based interactions in a world of ubiquitous computing
\cite{Marquardt:2011}.
%% we are nowadays
%% experiencing a new era of cooperation and interactions between humans.
%% The exponential proliferation of connected devices is
%% drastically impacting human experience and enhancing human
%% exchanges~\cite{Dearman:2008, Oh:2017, Sohn:2008,Harper08}. This evolution has opened new areas of research such as
%% \textit{proxemics} applied to ubiquitous computing
%% \cite{Marquardt:2011}. 
%% When evolving in their social space, users can today rely on
%% multiple kinds of computing devices that form a \textit{personal squad} of connected entities, or \squad as we shall call them in the remainder of the
%% paper.
Supported by their \squad, users
have come to expect always-on connectivity, along with advanced interaction
capabilities. For instance, as users bring
their devices close to one another, they expect to be able to initiate cross-device
interactions \textit{in situ} in order to share digital content
\cite{Oh:2017} between devices of their own \squad or with those of other users' \squad{}s. More broadly, users want to be able to share digital content seamlessly,
wherever the content, its sender, or its recipient might be located, from one
arbitrary device to another~\cite{Dearman:2008}, as they
would have a few years ago with an USB key or an email. 

This thirst for seamless sharing explains the rise and ongoing popularity
of personal sharing platforms such as Dropbox, Google Docs,
OneDrive, Amazon Drive, etc. These solutions allow users to share data with anyone, anywhere, at almost
anytime, between any devices. They achieve this feat by relying heavily on
\emph{remote cloud services}, which have now become a key enabler of ubiquitous ecosystems. 
% argument: cloud -> danger, 
%% From a technical perspective, one major trend to provide a seamless
%% sharing of digital contents among users' devices is to leverage on the
%% cloud, which acts as a key enabler for ubiquitous computing. Users' data
%% are pushed online, into the cloud with either Dropbox, Google Docs,
%% OneDrive, Amazon Drive, etc. to make it accessible anytime,
%% anywhere to the users' devices.
%% \commentFT{Removing the following, as seems to be a detour from main argument: \emph{In the last decades, a lot of research work has
%% emerged to promote the use of this offloading strategy to overcome scarce
%% resources of users' devices, and to provide an always-on availability
%% \cite{Zhang:2014,Gordon:2012,Chun:2011}. It paves the way of different kind
%% of clouds, such as mobile cloud computing, mobile-edge computing
%% and/or fog computing, that aim to use the nodes from the cloud
%% infrastructure that are the closest to end users
%% for offloading with an improved quality of service and user experience.}}
%% Hence, the cloud, and its variants constitute a perfect complement
%% to perform storage and computation on behalf of the resource
%% constrained devices.
Unfortunately, this reliance on the cloud comes at a strong price: 
%% users and their devices become totally tethered to the cloud ecosystem and its providers. As
%% a direct consequence,
users lose control over their own data, which is distributed over
multiple data centers, often in foreign jurisdictions, making it difficult for most users to grasp, review, and assess the privacy, and
confidentiality guarantees they can expect. In practice, user data hosted in the cloud is routinely harvested, often without explicit
consent, either by cloud providers or third parties, a sad situation that is only partially
mitigated by legal measures such as the GDPR.

Although data encryption is often proposed as a first line of defense for cloud-hosted data, it is unfortunately  insufficient~\cite{granick_2017, HooffLZZ15, HarnikPS10}. Metadata such as a user'
location and activity can still be tracked, including access
control histories, thus revealing what each user shared with whom and when. 
%  because it does not protect meta-data information
The exposure of metadata seriously weakens user privacy, as famously emphasized by a former NSA and CIA director: "\emph{We kill people based on metadata.}''~\cite{NaughtonTheGuardian2016} 
In this model, user devices, masquerading as allies, get turned into %% tracking and/or monitoring \squad{}s, and serve as
highly efficient monitoring tools spying on their owners. 

Constructing a system that provides fully anonymous content sharing while avoiding the above pitfalls is unfortunately highly
challenging. The last two decades have witnessed many attempts at building privacy-preserving and anonymous P2P data-sharing
networks~\cite{Clarke:2001,Gnunet2002,Freedman:2002,Nambiar:2006,Rennhard:2002}.
These solutions typically leverage \emph{onion
 routing}~\cite{Chaum:1981}, with some improvements to perform node
 discovery in a distributed manner, a step that is otherwise
centralized in traditional onion routing schemes. However, because of the high
churn typically experienced by P2P networks, the performance of these systems has been
discouraging~\cite{LeBlond:2013}, as churn causes routes to disappear and be
reconstructed frequently, which is particularly costly. By contrast, Tor
\cite{Tor}, one of the most popular anonymous networks,
trades lower anonymity guarantees for a much better
performance. \sonja{lower than who?} Tor's lower protection arises from
a partially centralized design that hinges on a set of \emph{directory
servers}, which are used to advertise trusted nodes that are available to bootstrap onion
routes. 

In this paper, we demonstrate how good performance can be reconciled
with strong privacy protection in file transfer systems. We present
\name{}, a novel fully decentralized, autonomous,
% approach that provides a 
and self-organizing file transfer service between the \squad{}s of different users that preserves user privacy
in terms of both data \emph{and} metadata. \commentFT{Moving to 3 ideas, to highlight proxemics in its own right} Our proposal hinges on three
key ideas: \emph{Firstly}, the emergence of \emph{\squad{}s}, the set of connected devices owned and used by one user, 
%% the recent advances of smartphones and other
%% mobile devices enable e-squads and it has thus
%% become possible to
makes it possible to apply P2P technology to one's own devices
alone, thereby reducing one's exposure to security and privacy threats
from third parties. \emph{Secondly}, we consider \emph{proxemics relationships} as a key enabler to initiate file
transfers. Physical proximity allow users to share
routing information out-of-band.
%, e.g. from their \squad. \commentFT{I did not get what 'from their \squad' mean.} 
In contrast to state-of-the-art P2P systems, users of \name agree out-of-band to share a file, but
do not keep it available for anyone else in the future. \emph{Thirdly}, we
exploit distributed machine learning techniques to predict the future
availability of user devices, and use this ability to implement a
\emph{probabilistic predictive onion routing} mechanism. The
predictive nature of our routing, exploited in a stateless way, allows
us to overcome the inherent fragility of pure P2P file transfer schemes. 

%% In this paper, we introduce \name{}, a novel approach to perform a decentralized, autonomous,
%% and self-organizing file sharing among users' devices while preserving the users’ privacy
%% in terms of data and metadata. We leverage on the latest works achieved
%% in p2p systems, privacy and anonymity. The originality of our approach come
%% from two key aspects. First, we consider proxemics relationships as a key enabler to initiate file
%% transfers. Intimate distance among users enable them to share
%% routing information out-of-band, e.g. from their
%% \squad. Compared to the state of the art on p2p systems, with \name,
%% users agree out-of-band to share a file but
%% do not keep it available for anyone else in the future. Second, we
%% rely on distributed machine learning technics to predict the
%% availability of nodes. As such, \name provides optimized onion routes; it avoids routes
%% reconstruction under churns.

Our contributions are as follows: 
\begin{itemize}

% \item We have designed \name, a novel fully decentralized, autonomous,
% and self-organizing file transfer service among users based on a novel
% concept named Stateless Probabilistic Onion Routes (SPOR), which combines onion routing with
% probabilistic node selection, informed by random peer sampling and
% predictions for node availability;
% \item We have extended Sphinx~\cite{Sphinx}, a compact and secure
%   message header format for mixnets, to allow for sets of nodes
%   instead of single nodes for onion routing layers, resulting in a
%   hybrid between source routing and ad-hoc node  selection that can
%   deal with unreliable nodes;
% \item  We have demonstrated the feasibility of our approach without
%   third-party storage, both in terms of security and performance . Even
%   suboptimal prediction results in successful file transfer with high
%   probability. 

  \item We provide the Sphinxes mix-header format which extends 
    Sphinx~\cite{Sphinx} to a source-routed limited random-walk protocol --- 
    \ie each layer of the onion route contains several alternative next nodes.

  \item We provide the \ac{SPOR} node selection strategy for routing which, given a secure 
    \ac{RPS} protocol and availability estimates, selects nodes to populate the 
    layers of Sphinxes to maximize the availability of the route --- this 
    yields reliable routes even with low-availability nodes.

  \item We provide a protocol for a device squad to estimate the availabilities 
    of its own devices, based on Markov models and \squad overlays.
    These availabilities can then be used with \ac{SPOR}.
   % \commentDaniel{We should add some more details, maybe.}

  \item We use \ac{SPOR} to design \name \sonja{removed for
      redundancy, rephrased: a novel, fully decentralized, 
    autonomous and self-organizing} as a privacy-preserving file transfer service among
  users without third-party storage.

\end{itemize}

In the following, in Section~\ref{system-model}, we state the
system and adversary models as well as the desired security and privacy properties of \name. Section~\ref{design}
introduces the design of our proposed solution. In
Sections~\ref{security-discussion} and \ref{Performance},
respectively, we evaluate the security and performance of our
approach. We set \name into context with related work in
Section~\ref{RelatedWork} and conclude with Section~\ref{conclusion}.

% Cashmere: resilient anonymous routing.
% Route fingerprinting in anonymous communications.
% Bridging and fingerprinting:
% Epistemic attacks on route selection
% Breaking the collusion detection mechanism of morphmix
% Denial of service or denial of security? How attacks
% on reliability can compromise anonymity
