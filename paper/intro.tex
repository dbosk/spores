\section{Introduction}

% What's the problem?
Following the Weiserian vision, nowadays, we are
evolving in a new era of cooperation and interactions among humans.
The exponential proliferation of devices connected to the Internet is
drastically impacting human experience and enhancing human
interactions \cite{Dearman:2008, Oh:2017, Sohn:2008,Harper08}. It further opens new area of research such as
\textit{proxemics} applied to ubiquitous computing
\cite{Marquardt:2011}. Users are evolving in their social space with 
multiple kinds of computing devices that form a \textit{squad} of users'
devices. Leveraging on the help of their devices squad, users increase their expectations in
terms of connectivity and interactions possibilities. For instance, as users bring
their devices close to one another, they expect to initiate cross
device interactions \textit{in situ} to share digital content
\cite{Oh:2017} among devices of their own or with devices 
from other users. More broadly, users expect to share digital content,
whatever the content location, from one
device to another in a seamless manner \cite{Dearman:2008}, as they
would have done in the old days with an USB key or with an email. 

From a technical perspective, one major trend to provide a seamless
sharing of digital contents among users' devices is to leverage on the
cloud, which acts as a key enabler for ubiquitous computing. Users' data
are pushed online, into the cloud with either Dropbox, Google Docs,
OneDrive, Amazon Drive, etc. to make it accessible anytime,
anywhere to the users' devices.  In the last decades, a lot of research work has
emerged to promote the use of this offloading strategy to overcome scarce
resources of users' devices, and to provide an always-on availability
\cite{Zhang:2014,Gordon:2012,Chun:2011}. It paves the way of different kind
of clouds, such as mobile cloud computing, mobile-edge computing
and/or fog computing, that aim to use the nodes from the cloud
infrastructure that are the closest to end users
for offloading with an improved quality of service and user experience.

Hence, the cloud, and its variants constitue a perfect complement
to perform storage and computation on behalf of the resource
constrained devices. However, this cloud reliance comes at a strong price:
devices become totally tethered to a cloud ecosystem and provider. As
a direct consequence, users lose the control over their own data as it is
stored and spanned into multiple foreign data centers. Hence, it becomes difficult at 
the legal level to assess the privacy, and
confidentiality delivered to the end users. Further, at the practical
level, users' data can be harvested on their behalf, and without their
consent, by either the cloud provider or third parties.
Data sent by the user can be encrypted, but it is not enough
\cite{granick_2017, HooffLZZ15}. As stated by a famous quotation from
NSA: "we kill people based on metadata''. Metadata such as the users'
location and activity can be still tracked, including the access
control history revealing what users share with whom and when.
Users’ devices become tracking and/or monitoring devices that turn to
be the best spy over users. 

In this paper, we introduce SPORE
