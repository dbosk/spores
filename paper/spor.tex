%!TEX root = spores.tex

\NewScheme{\SPOR}{SPOR}

\section{\Acf*{SPOR}}%
\label{SPOR}%
\label{sec:SPOR}%
\label{sec:message_passing}%

Say that Alice wants to send a message \(m\) to Bob.
Bob creates a reply header using \(\CreateReply\) and gives the output to Alice 
over an out-of-band channel\footnote{%
  There can be many reasons for this and many ways to solve it.
  \Eg \(m\) might be unavailable at the time.
  Signal creates \ac{DH} exponents in advance (before \(m\) is known) and 
  publishes on the Signal servers, so Alice would fetch one of Bob's when she 
  has a message to send.
  We will treat this in more detail later.
}.
At a later point, Alice attaches a message \(m\) to the reply header using 
\(\UseReply\), creates a forward header using \(\CreateFwd\) with the prepared 
reply header as parameter. \sonja{what about the crossover node and
  Alice's leg of the route?} 
Then she sends this packet to the first node, which processes it using 
\(\ProcessHeader\).
The node will then process the header and in turn forward the packet to the 
next node.
At some point the message will reach Bob who identifies the message as the 
one expected to come from Alice.
%(This process is illustrated in \cref{fig:file-transfer}.)
Now we will focus on how Alice and Bob choose the nodes that they provide to 
\(\CreateReply\) and \(\CreateFwd\).

%\begin{figure}
%  \includegraphics[width=\linewidth]{figures/file_transfer_v2.pdf}
%  \caption{\label{fig:file-transfer}%
%    A schematic of Alice and Bob sending a message using \name.
%    \ding{182} illustrates the layer of the headers that Alice and Bob create.
%    In \ding{183}, Alice and Bob exchange the headers out-of-band.
%    In \ding{184}, Alice and Bob use two \ac{SPOR} routes, one for messages and 
%    one for acknowledgements.
%  }
%\end{figure}

There are many uses of SphinxES, many criteria that can be used to select the 
nodes in each layer.
We will now describe the protocol~\(\SPOR\), which uses SphinxES to transfer 
messages from a source to a destination.
It chooses nodes to optimize for availability of each layer.
It also chooses them uniformly randomly from the entire set of 
devices~\(\devices\).

\((d, \pk_d, a_d)\gets \GetNode\): The algorithm~\(\GetNode\) returns a tuple 
\((d, \pk_d, a_d)\) of one selected node, where
\begin{itemize}
  \item \(d\) is the node's identifier (address);
  \item \(\pk_d\) is the node's public key;
  \item \(a_d\gets \avail(d)\) is the availability of device~\(d\), where the 
    function~\(\avail\colon \devices\to \interval{0}{1}\) maps a device to its 
    availability (0 is always offline, 1 is always online).
\end{itemize}
We note that the function~\(\avail\) does note necessarily have to return the 
exact availability of \(d\), \eg it can provide \(k\)-anonymity by mapping the 
\(d\)'s availability to the closest of some predefined availability values, 
such as \(\{0.25, 0.5, 0.75, 1\}\).

The \(\GetNode\) algorithm can be instantiated using \eg
Octopus~\cite{Octopus}, which would yield a near uniform distribution of nodes 
under adversarial conditions.
Another instantiation would be Tor's authoritative directory servers.
However, exactly how the \(\GetNode\) algorithm is implemented is not our 
concern at the moment.

When Bob runs \(\CreateReply\) he must supply a set of nodes for each layer of 
the route, \(L = \{L_0, \dotsc, L_\nu\}\), where \(|L_i| \leq w_L\) for \(0\leq 
i\leq \nu\).
The set \(L_\nu\) contains only one or more of Bob's devices.
For the other layers, \(L_i\) for \(i < \nu\), Bob will do the following:
Bob makes \((\nu-1)\cdot w_L\) queries to \(\GetNode\) to get enough nodes to 
fill all remaining layers.
Bob will distribute these nodes across all layers~\(L_i\) to maximize the 
availability~\[
  a_{L_i} = 1 - \prod_{d\in L_i} (1-a_d)
\] of every layer (or equivalently, minimizing the risk that every node in a 
layer is offline at the same time).

