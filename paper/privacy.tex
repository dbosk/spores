\section{Privacy}%
\label{Privacy}

The solution we consider is strictly \ac{P2P}.
From a privacy point-of-view, this is, at a first glance, good for privacy: 
there will be no central provider who can monitor everyone.
Unfortunately, with \iac{P2P} system, everything that was done in the Cloud 
before, must now be done in the public.
This means less privacy for Alice and Bob unless we carefully design the system 
from a privacy perspective, which is not easy~\cite{DevilInMetadata}.
Particular to our case, Alice and Bob will rely only on their own devices (\eg 
smartphones, laptops, tables \etc).
The online--offline patterns of the devices, which devices has certain files 
\etc will reveal information about Alice and Bob's behaviour.
We would like to limit how much of Alice's or Bob's behaviour can be learned by 
another party.

The first adversary model that we consider is friendly 
surveillance~\cite{FriendlySurveillance}.
Since Alice relies only on her own devices, whenever any of those devices must 
communicate with Bob (\eg to transfer a file), this will reveal information 
about Alice's behaviour to Bob.
(In the same way, this will also reveal information about Bob's behaviour to 
Alice.)
Alice's goal is to reveal as little of her behaviour to Bob as possible.
\commentDaniel{%
  What do we mean by behaviour and what is \enquote{little}? See issue \#15.%
}

We would also consider what Carol can learn about Alice's and Bob's behaviour.
Since this is \iac{P2P} system, Carol can participate and maybe Alice and Bob 
must use Carol in their communication.
\commentDaniel{%
  This should be the same problem as for Bob above, it's just that Carol doesn't 
  communicate with Alice directly.
  So still issue \#15.%
}

%The final adversary is a network-wide adversary.
%We know from \textcite{SystemsForAnonymousCommunication} that it is difficult to 
%protect oneself from an adversary who can monitor the entire network 
%(specifically the edges).
