\section{Related work}%
\label{RelatedWork}


\commentAL{Will need another pass.}
\paragraph*{Modeling devices' connection and disconnection} 
% To simulate users' devices participating in a distributed protocol, we build upon existing literature on two topics: 
% multi-device interaction, and churn in P2P networks. 
In our experiments, we simulate users' devices participating in a distributed protocol.
To make this simulation realistic, we based our models on existing literature. 

Firstly, fields studies provide \emph{usage and connectivity patterns of users' appliances}.
Karlson et al. \cite{Karlson_Meyers_Jacobs_Johns_Kane_2009} studied the compared usage of the laptop and smartphone of 16 information workers: 
they show that their participants use their smartphones very regularly throughout the day to do background tasks (\eg answering mails or checking their agenda).
Their laptop is online less often, and is dedicated to the most demanding tasks.
Wagner et al. \cite{Wagner_Rice_Beresford_2013} conducted a thorough data collection on Android devices via an application called Device Analyzer, that collected metrics on movement patterns, connectivity and energy consumption from 16,000 users.
\commentAL{Conclude: will we use their dataset? Why are these papers useful to us?}

Secondly, several research works point out the emerging trend of \emph{multi-device interaction}.
A 2012 Google-led study \cite{google2012} stated that the USA are nowadays ``a nation of multi-screeners''.
People switch their attention toward their phone, laptop, tablet or their television depending on the context.
M\"uller et al. \cite{Muller_Gove_Webb_Cheang_2015} specifically compared the usage patterns of phones against tablets.
Jokela et al. \cite{Jokela_Ojala_Olsson_2015} and \cite{google2012} also differentiated sequential and parallel usage of one's multiple devices.

Finally, research in P2P networks observed \cite{Stutzbach_Rejaie_2006} and modeled \cite{Yao_Leonard_Wang_Loguinov_2006} patterns of \emph{churn} (\ie participants moving in and out of a system).

We make use of these works to establish credible models of user behaviors to evaluate our protocol. 
Participants can own several devices: their smartphone is always on, but lacks processing power, while their laptop is rarely online, though it contains better hardware, etc. 


