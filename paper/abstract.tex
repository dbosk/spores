% What's the problem?
% Why is it a problem? Research gap left by other approaches?
% Why is it important? Why care?
% What's the approach? How to solve the problem?
% What's the findings? How was it evaluated, what are the results, limitations, 
% what remains to be done?
%
%abstract
%problem/topic
This paper presents privacy-preserving file sharing without relying on any third-party storage. 
%research gap
Centralized file-sharing systems, e.g.,  cloud storage services such as Dropbox, require trust in the service provider and can be monitored. Peer-to-peer (P2P) file-sharing relies on replicas on other people's machines. We are interested in minimizing the trust needed in other systems and preserving the users' privacy in terms of data and metadata, what they share with whom and when.
%why now
Nowadays, many people use multiple devices and it has thus become possible to benefit from P2P technology for one's own devices alone, thereby reducing one's exposure to security and privacy threats from third parties.
%approach
We combine onion routing with probabilistic node selection, informed by peer sampling and Hidden Markov Model predictions for node availability.
%findings/contributions:
We benefit from sender-receiver anonymity properties provided by onion routing to protect user privacy in decentralized file sharing. We show that even a relatively naive prediction model suffices to share files despite unreliable availability of individual nodes. The same model can potentially be applied for better-informed churn handling in other p2p systems. In the case of onion routing systems for anonymity, our solution allows the system to include unreliable nodes and take advantage of locality, neither of which is currently provided in TOR.

