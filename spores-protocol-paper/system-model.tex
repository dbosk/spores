% -*- tex-main-file: "contents.tex" compile-command: "pdflatex -halt-on-error spores.tex > /dev/null" ispell-dictionary: "american" -*-
%!TEX root = spores.tex


\section{System model, adversary model and desired security and privacy properties}%
\label{system-model}

%system model: explain e-squad, contrast with cloud

%adversary model: remove provider, compensate, results in same as in sphinxOR/Tor?  friends?

%properties: cam/lys: correctness, integrity, security (wrap, indistinguishability) - do not want friends to learn 1) which devices are ours 2)  their availability (assumption: RPS, not lookup)

In the following, we first describe how file transfer typically occurs
when users rely on a cloud storage provider. We use this simple use
case to highlight some of the desirable privacy properties of a file
transfer service, before mapping them into our fully decentralized
(a.k.a. P2P\footnote{We use the terms \emph{fully decentralized} and
  \emph{P2P} interchangeably, and preferentially the latter for brevity.})
context.
We finally consider the additional privacy properties
required to protect users from inferences, such as by network traffic
analysis, and state our adversary model.

\subsection{File transfer via a cloud storage provider}

When sharing a file with Bob through a cloud provider, Alice only
needs to upload this file from one of her devices to the cloud and
share a link to this file with Bob. Bob in turn can download this file
through the transmitted link onto one of his devices.

The use of the cloud provider, as a third party,
provides two main privacy properties, which we should preserve when
moving to a P2P setting:
\begin{itemize}
\item Alice and Bob are the only \emph{users} who learn about Alice sharing a 
  file with Bob.
\item The composition and behavior of Alice and Bob's \squad{}s remain
  hidden to each other, and to other users. I.e. Alice does not learn
  which devices Bob owns or uses, let alone how Bob uses them, and
  reciprocally.
\end{itemize} 
The cloud provider, however, has this information and any other
stemming from the users' interaction with the service, and can draw
inferences from metadata even when the data itself is encrypted.

\subsection{Moving to a P2P environment}

A private P2P file transfer solution should retain the privacy that
Alice and Bob already enjoy when a cloud provider functions as a
gatekeeper towards each other and other users.  Eschewing a cloud
provider, as we propose to do, removes not just the privacy risks
associated with a central brokering entity but also the benefits from
gatekeeping~\cite{DevilInMetadata} and therefore exposes users to friendly
surveillance~\cite{FriendlySurveillance} by other peers (\name
users including Alice and Bob themselves). In particular, in an
honest-but-curious setting, other users might collude to observe some
of the traffic, and gain knowledge regarding Alice's and Bob's
interactions, the devices they own, and how they use them.

\subsection{Basic assumptions}

We assume any pair of devices can communicate with each other as soon as one 
knows the address of the other (\eg using the Internet, possibly extended with 
NAT traversal techniques).
The devices owned by the same user (this user's \squad) know of each
other and their addresses.
We also assume there exists some protocol which allows them to synchronize 
their knowledge~\cite[\eg][]{luxey:cascade}.

We also assume that users can exchange some limited amounts of data 
out-of-band, \ie through some other channel --- \eg using \ac{NFC}, QR codes or 
simply send a link through instant messaging or email.

\subsection{Privacy properties and adversary model of \name}

\daniel{This section doesn't really say what the privacy properties are.}

\paragraph*{Privacy properties}

Friendly surveillance can be mitigated by ensuring security properties 
typically associated with onion routing.
\Textcite{CLOnionRouting} define security properties for onion routing as 
% We combine the requirements from
% the \ac{P2P} setting with those from onion routing for anonymity.
% Camenisch and Lysianskaya \cite{CLOnionRouting} define security properties 
% for onion routing:
\textit{onion-correctness, integrity}, and \textit{security}. %(wrap resistance,
                                %indistinguishability)
Informally, \textit{onion-correctness} means that if an onion is formed
correctly, processed by the correct routers in the correct order, then
the correct message is received by the last router. \textit{Onion-integrity}
means that even for an onion created by an adversary, the path length
is bounded by $N$. \textit{Onion-security} means that an adversary sending an
 onion to a router as a challenge
 and able to see the response (the outcome of how the
onion is processed), cannot distinguish between whether that onion
corresponds to a particular message and route or not. This is also
true for a \emph{re-wrapped} onion, \ie a challenge (onion) wrapped in
another layer. The formal definitions are in~\cite{CLOnionRouting} along
with a theorem that states that an onion routing scheme satisfying
these three properties, combined with secure point-to-point channels,
is \acadv{UC} secure, \ie the protocol remains
secure even if arbitrarily composed with other instances of the same or other protocols.
These properties cover the concerns about
revealing to others that Alice shares a file with Bob in the first
place. In \name, we additionally derive the property of
\emph{device-unlinkability} from the \ac{P2P} requirements introduced above, meaning that
using \name neither reveals the \squad a device belongs to, nor its
owner. Thus, no information on the user's behavior (e.g. being online)
can be infered through \name. 
\sonja{insert more formal definition
  here, and perhaps explanation that we need it because we introduce
  availability information}


 % \commentFT{Removed, as did not fit with
 %  the flow, but probably needed somewhere: `We give the definitions
 %  for the security and privacy properties of \name in
 %  Section~\ref{security-discussion}.'} \sonja{better here. Need to define
 %  device unlinkability, the rest is covered by Camenisch.}

\paragraph*{Adversary model}

To assess our approach, we use the
adversary model assumed by Tor and \ac{Sphinx}:
the adversary has complete control over some part of the network as
well as some of the nodes. Like them, we do not consider a passive global network
adversary, against which it is notoriously difficult to achieve
security~\cite{SystemsForAnonymousCommunication}.
